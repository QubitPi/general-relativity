\documentclass{book}

\usepackage{fontspec}
\setmainfont{Ubuntu}

\usepackage[left=1in, right=1in, top=1in, bottom=1in]{geometry}

\usepackage{fancyhdr}
\pagestyle{fancy}
\lhead{
    \includegraphics[scale=0.05]{github.png}
    \href{https://github.com/QubitPi/general-relativity}{Study Notes}
}
\chead{General Relativity}
\rhead{
    \href{https://github.com/QubitPi}{QubitPi}
    \includegraphics[scale=0.05]{logo-8th-version.png}
}
\lfoot{}
\cfoot{}
\rfoot{\thepage}
\renewcommand{\headrulewidth}{0.4pt}
\renewcommand{\footrulewidth}{0.4pt}

\usepackage{hyperref}
\hypersetup{
    colorlinks=true,
    linkcolor=blue,
    anchorcolor=blue,
    urlcolor=blue
}

\usepackage{graphicx}
\usepackage{float}
\graphicspath{ {./img/} }

\usepackage{tikz}
\usepackage[most]{tcolorbox}
\tcbuselibrary{skins}
\tcbuselibrary{raster}

\usepackage{varioref}
\usepackage{cleveref}

\definecolor{myblue}{RGB}{0,163,243}
\NewTcbTheorem[auto counter, crefname={definition}{definitions}]
{Definition}{Definition}{
    fonttitle=\bfseries\upshape,
    fontupper=\slshape,
    arc=0mm,
    colback=myblue!20,
    colframe=myblue
}{def}

\newtcbtheorem[]{mytheorem}{Theorem}
{theorem style=plain apart,label type=theorem,enhanced,frame hidden,
boxrule=2mm,titlerule=0mm,toptitle=1mm,bottomtitle=1mm,
fonttitle=\bfseries\large,fontupper=\normalsize,
coltitle=green!35!black,colbacktitle=green!15!white,
colback=green!50!yellow!15!white,borderline={1pt}{0pt}{green!25!blue},
}{theo}

\newtcbox{\roundinlinebox}[1][red]{
    on line,
    arc=7pt,
    colback=#1!10!white,
    colframe=#1!50!white,
    before upper={\rule[-3pt]{0pt}{10pt}},
    boxrule=1pt,
    boxsep=0pt,left=6pt,right=6pt,top=2pt,bottom=2pt
}

% https://tex.stackexchange.com/a/307436/277953
\newtcolorbox{marker}[1][]{enhanced,
before skip=2mm,after skip=3mm,
boxrule=0.4pt,left=5mm,right=2mm,top=1mm,bottom=1mm,
colback=yellow!50,
colframe=yellow!20!black,
sharp corners,rounded corners=southeast,arc is angular,arc=3mm,
underlay={%
    \path[fill=tcbcolback!80!black] ([yshift=3mm]interior.south east)--++(-0.4,-0.1)--++(0.1,-0.2);
    \path[draw=tcbcolframe,shorten <=-0.05mm,shorten >=-0.05mm] ([yshift=3mm]interior.south east)--++(-0.4,-0.1)--++(0.1,-0.2);
    \path[fill=yellow!50!black,draw=none] (interior.south west) rectangle node[white]{\Huge\bfseries !} ([xshift=4mm]interior.north west);
},
drop fuzzy shadow,#1}

\usepackage{varwidth}
\usepackage{amsfonts}
\usepackage{mathrsfs}
\usepackage{parskip}

\setlength{\parindent}{0pt}
