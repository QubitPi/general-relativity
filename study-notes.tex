\documentclass{book}

\usepackage{fontspec}
\setmainfont{Ubuntu}

\usepackage[left=1in, right=1in, top=1in, bottom=1in]{geometry}

\usepackage{fancyhdr}
\pagestyle{fancy}
\lhead{
    \includegraphics[scale=0.05]{github.png}
    \href{https://github.com/QubitPi/general-relativity}{Study Notes}
}
\chead{General Relativity}
\rhead{
    \href{https://github.com/QubitPi}{QubitPi}
    \includegraphics[scale=0.05]{logo-8th-version.png}
}
\lfoot{}
\cfoot{}
\rfoot{\thepage}
\renewcommand{\headrulewidth}{0.4pt}
\renewcommand{\footrulewidth}{0.4pt}

\usepackage{hyperref}
\hypersetup{
    colorlinks=true,
    linkcolor=blue,
    anchorcolor=blue,
    urlcolor=blue
}

\usepackage{graphicx}
\usepackage{float}
\graphicspath{ {./img/} }

\usepackage{tikz}
\usepackage[most]{tcolorbox}
\tcbuselibrary{skins}
\tcbuselibrary{raster}

\usepackage{varioref}
\usepackage{cleveref}

\definecolor{myblue}{RGB}{0,163,243}
\NewTcbTheorem[auto counter, crefname={definition}{definitions}]
{Definition}{Definition}{
    fonttitle=\bfseries\upshape,
    fontupper=\slshape,
    arc=0mm,
    colback=myblue!20,
    colframe=myblue
}{def}

\newtcbtheorem[]{mytheorem}{Theorem}
{theorem style=plain apart,label type=theorem,enhanced,frame hidden,
boxrule=2mm,titlerule=0mm,toptitle=1mm,bottomtitle=1mm,
fonttitle=\bfseries\large,fontupper=\normalsize,
coltitle=green!35!black,colbacktitle=green!15!white,
colback=green!50!yellow!15!white,borderline={1pt}{0pt}{green!25!blue},
}{theo}

\newtcbox{\roundinlinebox}[1][red]{
    on line,
    arc=7pt,
    colback=#1!10!white,
    colframe=#1!50!white,
    before upper={\rule[-3pt]{0pt}{10pt}},
    boxrule=1pt,
    boxsep=0pt,left=6pt,right=6pt,top=2pt,bottom=2pt
}

% https://tex.stackexchange.com/a/307436/277953
\newtcolorbox{marker}[1][]{enhanced,
before skip=2mm,after skip=3mm,
boxrule=0.4pt,left=5mm,right=2mm,top=1mm,bottom=1mm,
colback=yellow!50,
colframe=yellow!20!black,
sharp corners,rounded corners=southeast,arc is angular,arc=3mm,
underlay={%
    \path[fill=tcbcolback!80!black] ([yshift=3mm]interior.south east)--++(-0.4,-0.1)--++(0.1,-0.2);
    \path[draw=tcbcolframe,shorten <=-0.05mm,shorten >=-0.05mm] ([yshift=3mm]interior.south east)--++(-0.4,-0.1)--++(0.1,-0.2);
    \path[fill=yellow!50!black,draw=none] (interior.south west) rectangle node[white]{\Huge\bfseries !} ([xshift=4mm]interior.north west);
},
drop fuzzy shadow,#1}

\usepackage{varwidth}
\usepackage{amsfonts}
\usepackage{mathrsfs}
\usepackage{parskip}

\setlength{\parindent}{0pt}


\begin{document}

    \part{A First Course in General Relativity\cite{first-course-on-gr}}

    \chapter{Special Relativity}

    \section*{On ``Principle of relativity (Galileo)"}

    \subsection*{Galilean invariance}

    \href{https://en.wikipedia.org/wiki/Newton\%27s_laws_of_motion}{Newton's laws of motion} hold in all frames related
    to one another by a \href{https://en.wikipedia.org/wiki/Galilean\_transformation}{Galilean transformation}. In
    other words, all frames related to one another by such a transformation are inertial (meaning, Newton's equation of
    motion is valid in these frames).\cite{galilean-invariance} The proof has been given by the book on page 2.

    \section*{1.5 - Construction of the coordinates used by another observer}

    \subsection*{Why would the tangent of the angle is the speed in Fig. 1.2?}

    Suppose $\mathscr{O}$ and $\mathscr{\bar{O}}$ both start out at the same position where $\mathscr{\bar{O}}$
    moves along the $x$ at some speed. After $t_1$, observer $\mathscr{O}$ sees $\mathscr{\bar{O}}$ at position $x_1$:

    \[ \mathscr{\bar{O}}_1 = (x_1, t_1) \]

    Observer  $\mathscr{\bar{O}}$, however, still sees themself at $x = 0$:

    \[ \mathscr{\bar{O}}_1 = (0, t_1) \]

    By definition where ``$\bar{t}$ is the locus of events at constant $\bar{x} = 0$", $\bar{t}$ is the straight line
    that passes the origin and the $(x_1, t_1)$:

    \begin{tikzpicture}[scale=1.5]
        \draw[->] (-3,0) node[left] (w) {}--(5,0) node[right] (x) {x};
        \draw[->] (0,-3) node[below] (s) {}--(0,5) node[above] (t) {t};
        \draw[line width=.5pt] (.25,-.25) rectangle (-.25,.25) node (o) {};
        \draw[line width=.5] (o)--(135:1) node[above] () {$\mathscr{\bar{O}}_0$, $\mathscr{O}_0$};

        \draw[black,line width=2pt,->] (0,0)--(4,4) node[above] (t-bar) {{$\bar{t}$}};

        \draw[dashed,line width=.5] (3,0) node[below] (x1) {$x_1$} --(3,3);
        \draw[dashed,line width=.5] (0,3) node[left] (t1) {$t_1$} --(3,3);
        \draw[line width=.5pt] (2.75,3.25) rectangle (3.25,2.75) node (f) {};
        \draw (f) node[right] {$(x_1, t_1)$};

        \draw [blue,thick,domain=225:270] plot ({3 + cos(\x)}, {3 + sin(\x)}) node (speed) {};
        \draw (2.5, 1.8) node[right] (speed) {Tangent of this angle is $\bar{v}$};
    \end{tikzpicture}

    \bibliographystyle{plain}
    \nocite{*}
    \bibliography{refs}

\end{document}
