\section{Gauss's Law for Electic Fields}

There are two kinds of electric field:

\begin{enumerate}
    \item the \textit{electrostatic} field produced by electric charge
    \item the \textit{induced} electric field produced by a changing magnetic field
\end{enumerate}

Gauss's law for electric fields deals with the electrostatic field. It relates the spatial behavior of the electrostatic
field to the charge distribution that produces it. The integral form is generally written like this:

\subsection{The Electric Field}

To understand Gauss's law, we first have to understand the concept of the electric field. In some physics and
engineering books, no direct definition of the electric field is given; instead we see a statement that an electric
field is ``said to exist'' in any region in which electrical forces act. But what exactly is an electric field? This
question has deep philosophical significance and it is not easy to
answer\footnote{\cite[p.~3]{student-maxwell-equations}}. It traces all the way back to a person named Michael Faraday
who is believed to discover the concept of an electric field.

While Faraday did not develop a complete mathematical description of the electric field, his concept laid the groundwork
for later scientists to quantify the electric field using mathematical equations. The book by Michael Faraday that
introduced the concept of the electric field is called \textit{Experimental Researches in Electricity}, particularly in
its \textbf{Eleventh Series}, a chapter of the book.

Faraday conducted an experiment using two long coils. Let's call them coils $A$ and $B$. Coil $A$ was connected to a
battery source while coil $B$ was connected to a falavanometer, which measures the current in $B$. He discovered that,
when coils were long enough and battery source was strong enough, the falavanometer signaled a slight current passing
through coil $B$ at the moment of connecting coil $A$ the battery. In addition, when coil $A$ and the battery was
disconnected , the same amount of currrent in coil $B$ was detected again, but this time the current was in the opposite
direction.\footnote{\textit{Experimental Researches In Electricity, Vol. 1}, Faraday, Michael, 6 - 11}
