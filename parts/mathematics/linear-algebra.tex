\section{Metric Spaces}

\begin{tcolorbox}[enhanced,arc=3mm,boxrule=1.5mm,
    frame hidden,colback=blue!10!white,
    borderline={1mm}{0mm}{blue,dotted} ]
    The materials of this section refers largely to
    \href{https://trello.com/c/3EPccNTa}{Introduction to Topology and Functional Analysis by Goerge F. Simmons}
\end{tcolorbox}

\begin{Definition}{Metric Space}{metric-space}
    Let $X$ be a non-empty set. A \textbf{metric}\footnote{\href{https://trello.com/c/3EPccNTa}{Introduction to Topology and Functional Analysis by Goerge F. Simmons}, p.81}
    on $X$ is a real function $d$ of ordered pairs of elements of $X$ which satisfies the following conditions

    \begin{itemize}
        \item $d(x, y) \ge 0$, and $d(x, y) = 0 \iff x = y$
        \item $d(x, y) = d(y, x)$ (symmetry)
        \item $d(x, y) \le d(x, z) + d(z, y)$ (the triangle inequality)
    \end{itemize}
\end{Definition}

$d(x, y)$ is called the \textit{distance} between $x$ and $y$. Thus a metric space consists of 2 objects:

\begin{enumerate}
    \item a non-empty set $X$, and
    \item a metric $d$ on $X$
\end{enumerate}

Now let $X$ be a metric space with metric $d$, and let

\begin{equation}
    \{ x_n \} = \{ x_1, x_2, \cdots, x_n, \cdot \}
\end{equation}

be a sequence of points in $X$. We say that $\{ x_n \}$ is \textbf{convergent} if there exists a point $x$ in $X$ such
that either

\begin{enumerate}
    \item for each $\epsilon > 0$, there exists a positive integer $n_0$ such that $n \ge n_0 \Rightarrow d(x_n, x) < \epsilon$,
          or, equivalently,
    \item for each open sphere $S_{\epsilon}(x)$ centered on $x$, there exists a positive integer $n_0$ such that $x_n$
          is in $S_{\epsilon}(x)$ for all $n \ge n_0$
\end{enumerate}

We say in this case that $x_n$ converges to $x$ and $x$ is called the \textit{limit} of the sequence $\{ x_n \}$ and we
sometimes write $x_n \rightarrow x$ in the form of

\[ \lim x_n = x \]

Every convergent sequence $\{ x_n \}$ has the following property: for each $\epsilon > 0$, there eixsts a positive
integer $n_0$ such that $m, n \ge n_0 \Rightarrow d(x_m, x_n) < \epsilon$



\begin{Definition}{Linear Space}{linear-space}
    Let $L$ be a non-empty set, and assume that each pair of elements $x$ and $y$ in $L$ can be combined by a process
    called \textbf{addition} to yield an element $z$ in $L$ denoted by $z = x + y$. Assume also that this operation of
    addition satisfies the following conditions:

    \begin{itemize}
        \item $x + y = y + x$
        \item $x + (y + z) = (x + y) + z$
        \item There exists in $L$ a unique element, denoted by $0$ and called the \textbf{zero element}, or the origin,
              such that $x + 0 = x$ for every $x$
        \item To each element $x$ in $L$ there corresponds a unique element in $L$, denoted by $-x$ and called the
              negative of $x$, such that $x + (-x) = 0$.
    \end{itemize}

    We adopt the device of referring to the system of real numbers or to the system of complex numbers as the
    \textbf{scalers}. We now assume that each scalar $\alpha$ and each element $x$ in $L$ can be combined by a process
    called \textbf{scalar multiplication} to yield an element $y$ in $L$ denoted by $y = \alpha x$ in such a way that

    \begin{itemize}
        \item $\alpha(x + y) = \alpha x + \alpha y$
        \item $(\alpha + \beta)x = \alpha x + \beta x$
        \item $(\alpha\beta)x = \alpha(\beta x)$
        \item $1 \cdot x = x$
    \end{itemize}

    The albegraic system $L$ defined by these operations and axioms is called a
    \textbf{linear space}\footnote{\href{https://trello.com/c/3EPccNTa}{Introduction to Topology and Functional Analysis by Goerge F. Simmons}, p.81}
\end{Definition}

Depending on the numbers admitted as scalars (only the real numbers, or all the complex numbers), we distinguish when
necessary between \textit{real linear spaces} and \textit{complex linear spaces}. A linear space is often called a
\hyperlink{vector-space}{\textit{vector space}} and its elements are spoken of as \textit{vectors}.

\begin{Definition}{Normed Linear Space}{normed-linear-space}
    A \textbf{Normed Linear Space}\footnote{\href{https://trello.com/c/3EPccNTa}{Introduction to Topology and Functional Analysis by Goerge F. Simmons}, p.81}
    is a linear space on which there is a \textbf{norm} defined, i.e. a function which assigns to each element $x$ in
    the space a real number $\Vert x \Vert$ in such a manner that

    \begin{itemize}
        \item $\Vert x \Vert \ge 0$, and $\Vert x \Vert = 0 \iff x = 0$
        \item $\Vert x + y \Vert \le \Vert x \Vert + \Vert y \Vert$
        \item $\Vert \alpha x \Vert = \vert \alpha \vert \Vert x \Vert$
    \end{itemize}
\end{Definition}

Intuitively, a normed linear space is simply a linear space in which a notion of the distance from an arbitrary element
to origin is defined.

\section{Vector Spaces and Subspaces}

\begin{Definition}{Vector Space \phantomsection\hypertarget{vector-space}}{vector-space}
    The space $\mathbb{R}^n$ consists of all column vectors $\boldsymbol{v}$ with $n$ components\footnote{\href{https://trello.com/c/qHJeDNkU}{Introduction to Linear Algebra, Strang, 4th Edition, 2009}, p. 120}
\end{Definition}

The components of $\boldsymbol{v}$ are real numbers, which is the reason for the letter $\mathbb{R}$. A vector whose $n$
components are complex numbers lies in the space $\mathbb{C}^n$

\begin{Definition}{Subspace}{subspace}
    A \textbf{subspace}\footnote{\href{https://trello.com/c/qHJeDNkU}{Introduction to Linear Algebra, Strang, 4th Edition, 2009}, p. 122}
    of a \hyperlink{vector-space}{vector space} is a set of vectors (including $\boldsymbol{0}$) that satisfies 2
    requirements: If $\boldsymbol{v}$ and $\boldsymbol{w}$ are vectors in the subspace and $c$ any scalar, then

    \begin{itemize}
        \item $\boldsymbol{v} + \boldsymbol{w}$ is in the subspace
        \item $c\boldsymbol{v}$ is in the subspace
    \end{itemize}

    In other words, the set of vectors is ``\textbf{closed}" under addition and multiplication - \textbf{all linear combinations
    stay in the subspace}

\end{Definition}

Intuitively, we can visualize a subspace in the 3-dimensional space $\mathbb{R}^3$. Choose a plane through the
origin $(0, 0, 0)$. That plane is a vector space in its own right. If we add two vectors in the plane, their sum is in
the plane; if we multiply an in-plane vector by $2$ or $-5$, it is still in the plane. This plane is a vector space
\textbf{inside $\mathbb{R}^3$} or is a subspace of the full vector space $\mathbb{R}^3$

\begin{Definition}{Column Space \phantomsection\hypertarget{column-space}}{column-space}
    The \textbf{column space}, $C(A)$, consists of all linear combinations of the columns, i.e. the combinations of all
    possible vectors $A\boldsymbol{x}$

    The subspece $C(A)$ is the ``\textbf{span}" of matrix $A$
\end{Definition}

\begin{Definition}{Basis for a Vector Space}{basis}
    A \textbf{basis}\footnote{\href{https://trello.com/c/qHJeDNkU}{Introduction to Linear Algebra, Strang, 4th Edition, 2009}, p. 172} for a vector space is a sequence of vectors with 2 properties

    \begin{itemize}
        \item The basis vectors are linearly independent, and
        \item they \hyperlink{column-space}{span} the space
    \end{itemize}
\end{Definition}

For a basis $\{\boldsymbol{v_1}, \ldots, \boldsymbol{v_d}\}$ of $V$

\begin{Definition}{Bilinear Form \phantomsection\hypertarget{bilinear-form}}{bilinear-form}
    Let $F$ be a field and $V$ a vector space over $F$. A \textit{bilinear form} on $V$ is a function
    $B: V \times V \rightarrow F$ that is linear in each variable when the other one is fixed. That is

    \begin{align}
        B(v + v', w) &= B(v, w) + B(v', w) \\
        B(cv, w)     &= cB(v, w)
    \end{align}

    for all $v, v', w \in V$ and $c \in F$, and

    \begin{align}
        B(v, w + w') &= B(v, w) + B(v, w') \\
        B(v, cw)     &= cB(v, w)
    \end{align}

    for all $v, v, w' \in V$ and $c \in F$

    We call $B$ \textit{symmetric} when

    \begin{equation}
        B(v, w) = B(w, v)
    \end{equation}

    for all $v, w \in V$\footnote{\href{https://kconrad.math.uconn.edu/blurbs/linmultialg/bilinearform.pdf}{Keith Conrad}}

    \begin{tcbraster}[
        raster columns=2,
        raster equal height
    ]
        \begin{Definition}{Linear Transformation}{linear-transformation}
            The transformation is \textbf{linear} if it meets these requirements for all $\boldsymbol{v}$ and $\boldsymbol{w}$\footnote{\href{https://trello.com/c/qHJeDNkU}{Introduction to Linear Algebra, Strang, 4th Edition, 2009}, p. 375}:

            \begin{equation}
                T(c\boldsymbol{v} + d\boldsymbol{w}) = cT(\boldsymbol{v}) + dT(\boldsymbol{w})
            \end{equation}

            for all $c$ and $d$
        \end{Definition}
        \begin{Definition}{Field}{field}
            A field is a set of elements in which a pair of operations called multiplication and addition is defined analogous
            to the operations of multiplication and addition in the real number system (which is itself an example of a
            field)\footnote{\href{https://projecteuclid.org/ebooks/notre-dame-mathematical-lectures/Chapter-I-Linear-Algebra/chapter/Chapter-I-Linear-Algebra/ndml/1175197044}{A. Fields, Chapter I: Linear Algebra}, Galois Theory: Lectures Delivered at the University of Notre Dame, Project Euclid}
        \end{Definition}
    \end{tcbraster}

    \begin{Definition}{Group}{group}
        A group\footnote{\href{https://www.maths.gla.ac.uk/~mwemyss/teaching/3alg1-7.pdf}{Introduction to Group Theory, Michael Wemyss}} is a non-empty set $G$ together with a rule that assigns to each pair $g$ and $h$ of elements of $G$
        and element $g * h$ such that
        \begin{itemize}
            \item[] \includegraphics[width=0.05\textwidth]{嘟嘟可.png}  $g * h \in G$, which we say $G$ is \textbf{closed} under $*$
            \item[] \includegraphics[width=0.05\textwidth]{嘟嘟可.png} $g * (h * k) = (g * h) * k$ for all $g, h, k \in G$, which we call $*$ being
            \textbf{associative}
            \item[] \includegraphics[width=0.05\textwidth]{嘟嘟可.png} There exists an \textbf{identity} element $e \in G$ such that $e * g = g * e$ for all
            $g \in G$
            \item[] \includegraphics[width=0.05\textwidth]{嘟嘟可.png} Every element $g \in G$ has an \textbf{inverse} $g^{-1}$ such that
            $g * g^{-1} = g^{-1} * g = e$
        \end{itemize}
    \end{Definition}

    \begin{Definition}{Vector Space (Field Theory)}{vector-space-field}
        If $V$ is an additive abelian group with elements $A, B, \ldots$, $F$ a field with elements $a, b, \dots$, and if
        for each $a \in F$ and $A \in V$ the product $aA$ denotes element of $V$, then $V$ is called a
        \textbf{(left) vector space over F} if the following assumptions hold:

        \begin{align}
            a(A + B) &= aA + aB \\
            (a + b)A &= aA + bA \\
            a(bA) &= (ab)A \\
            1A &= A \\
        \end{align}
    \end{Definition}
\end{Definition}